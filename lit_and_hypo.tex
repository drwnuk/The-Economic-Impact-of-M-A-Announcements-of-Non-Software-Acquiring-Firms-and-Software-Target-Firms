\section{Background, Related Work and Hypotheses}\label{Sec2}



M\&A refers to company takeovers, where a merger is about combining two firms with complementary assets and size, into a new constructed entity. Meanwhile, an acquisition refers to one firm purchasing another firm (i.e. a target firm), which then will be owned by the acquiring firm \cite{vonGersdorff}. The similarities between each process stage for both mergers and acquisitions, indicates that M\&A can be performed with almost the same integration procedures. There are four phases: (i) Consideration - the acquiring and target firm has the time to reflect on why they need to perform a takeover, and envision the future outcome together, (ii) Negotiation - the legal proceeding, where the involved firms will discuss the transaction terms and determines if the takeover will be a merger or an acquisition, (iii) the transaction deal - mostly consist of lots of paperwork (e.g. obtaining approval from government agencies, examine the agreements/contracts, summarize financial statements, etc.), (iv) finalizing the takeover, which in some cases can take several years to complete. This work focuses more on the economic impact generated from M\&A announcements rather than the M\&A transaction itself, the author highly recommend that readers should instead refer to Calipha et al. \cite{calipha2010mergers}, Popp \cite{popp2013mergers} and Schief et al. \cite{schief2013mergers}.

\subsection{Wealth effect around the event day}

From financial perspective the effects of M\&As have been measured using accounting data \cite{Healy,Ghosh} or market data \cite{agrawal1992post}. In the first approach some of the studies prove that M\&As translate into improved operational efficiency, \cite{Healy} other suggest it to be just the opposite \cite{Kruse}, and some do not point to any relationship between the M\&A transaction and their financial performance \cite{Ghosh}. In the second approach, that the articles corresponds literature do not give consistent results, e.g. Yilmaz \& Tanyeri \cite{Yilmaz} has reported positive abnormal returns for both acquiring and target firms whereas Campa \& Hernando \cite{Campa} has reported positive abnormal returns only for shareholders of target firms, and negative abnormal returns for acquiring firms. Nevertheless, these findings of mergers being beneficial only for shareholders of target company seems to be dominant in the literature. 

Also, an interesting point to note, is that acquiring firms tends to have higher risk of receiving negative abnormal returns compared to target firms. As a result of the conflicting claims, many academic research has provide different facts regarding if the effect of M\&A announcements will create or destroy value for a firm's shareholders which has been debated and analyzed within different context area with various cases in the past \cite{Mateev}. In the data sample of Moeller \cite{MOELLER2005533} it was proven that companies paid over USD 3.000 bn for takeovers while the shareholders lost because of these decisions USD 218 bn in market capitalization overall. That is way the M\&As among some researchers are considered as destroying value transactions \cite{bieshaar2001deals,sundersanam2006} what lead to a conclusion that companies behave irrational taking over other company. We investigate these returns for software and non-software M\&As in the following hypothesis.
\\
\\
\textit{H1: Shareholders of target and acquiring companies experience positive abnormal returns around announcement date}
\\

M\&A researchers have been examining different variables that could possibly impact observed abnormal returns. We decided to focus our research on the two main aspects that were not examined when taking into account only targets from software industry.

\subsection{Geographical effect}

Longterm success rate of M\&A in the software market remains low, where in most cases the synergy between companies and the financial outcomes indicates to be less than spectacular \cite{McCarthy}. The software industry consists of a large and varied market, where the competition between firms extends across the globe. Also, the line-of-business operations in the software market are not limited to any human resource or customer relationship, therefore it is quite usual for firms to perform cross-border M\&As especially in software industry.  Having a cross-border alliance means that the resources are shared in a joint headquarter, and is commonly stationed in a different country \cite{ZHU2015556}. This may leads to distributed development challenges, extra costs with on-boarding and other unanticipated issues associated with global software engineering \cite{4221620,vsmite2010empirical}. 

It might not always be the case that an acquiring firm will find a suitable target, that operates within their local geographic boundaries. Consequently, a buyer is then forced to comply, and need to step out of their comfort zone \cite{Huysman}. Country effect can namely be divided into two categories, i.e. domestic M\&A announcement and cross-border M\&A announcement. Which refers to whether if the target firm is located in the same or another country.

Globalization lowers entry barriers for new countries, thus the growth in cross-border M\&A transactions has become more evidently \cite{Mateev} . Since the world is becoming more integrated relative to technology and corporate/economic activities, the purpose behind cross-border M\&As are in most cases about coping with the fiercely international competition that is growing every passing year. Consequently, cross-border partnerships deals accounts for approximately one-third of worldwide M\&A activities, and this figure will certainly keep increasing in the future \cite{erel}. In software industry we observed same pattern over last decades where only 7\% of deals were international in 80s wile 30\% in 10s current age \cite{quo vadis}.

Based on past event studies, there has been different reports regarding cross-border M\&As. For example, while Erel \cite{erel} find acquiring abnormal return to be significantly higher with cross-border target firms (preferably located in a nearby country), other researchers observe a contradictory result \cite{BOUBAKRI20121,andrade}. Yet both of these results are not in line with \cite{Mateev}, which reported no significant difference between domestic or cross-border target firms. Due to the potential benefits with cross-border compared to domestic M\&As \cite{MateevAndovan}, we state the following hypothesis. 
\\
\\
\textit{H2: Shareholders of targets and acquirers experience greater significant economic impact during cross-border M\&As 
compared to domestic}
\\
\subsection{Industry Relatedness}

%Sometime there might be a necessity for a firm to acquire another firm from other industries than their own, due to lack of resources or capabilities (e.g. a software firm may acquire a non-software firm, or vice versa). This strategy is appealing for many non-software companies that transform their daily operations based on digitalization. Merging with a software company, a non-software company gains knowledge that reinforces the absorptive capacity, i.e. ability to recognize the value of new information, and utilize the newly obtained technology or knowledge. -> this is aim of our study, its not the place here

The industry relatedness refers to if whether an acquiring and a target firm operates in the same industry or in different industries. Industry is a prominent factor to examine as the cause of impact on abnormal return during M\&A \cite{ellwanger2015}. A wide range of past event studies have demonstrated similar consensus results in regard to industry relatedness. For example, \cite{scanlon1989,Bley2003,lim2016} find acquiring firms' abnormal return to be significantly higher when a target firm is operating in the same related industry. The reason behind why an acquiring firm is expected to receive a high abnormal return, is that firms in related industry creates better synergy effect and more opportunities to gain market shares \cite{scanlon1989impacts}. The study by  Pierre-Majorique \& Shinkyu \cite{Lger2005NetworkEA} that involved software to software M\&As brings only partially consistent findings. Since our research focuses only on transactions between non software and software companies they are by default only cross section deals. However, close relationship between industries exist and deal between software and software related industries should yield better results than those between unrelated companies.
\todo[inline]{some connection between presented theory and our problem
WNUK: Maybe mention digitalization and that traditional industries are getting more software-intensive like cars, trucks etc.}

By assuming that it is easier to transfer skills and knowledge between firms within the same industry, will indirectly suggest that firms from different industries will underperform. This conjecture is in line with Amewu \& Alagidede \cite{amewu}, who argue that, acquiring a target firm from an unrelated industry will not earn any significant abnormal return for the acquiring firms' shareholders. Gregor and MacCorriston \cite{Gregory} prove that these acquisitions have a negative impact on the wealth of shareholders, explaining  that managers do not know the industry of the company being acquired. Martynova \& Renneboog (2006) \cite{martynova2006mergers} disagreed and concluded that conglomerate mergers, i.e. mergers between firms from different industries will generate higher abnormal returns for target firms' shareholders. We are here also interested in understanding what domains are challenging for software companies to enter via M\&A and why. 
\tood[inline]{WNUK: Here we can write some stories based on the list of companies}
\\
\\
\textit{H3: Target and acquirer will experience greater significant economic impact from the M\&A announcement if the target firm operates in a close related industry}\\